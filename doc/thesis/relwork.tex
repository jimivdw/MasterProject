\chapter{Related work}
\chlab{relwork}

Quite a lot of literature is available that relates to interaction design for systems like the tool for fragment-based molecule parameterisation that we discuss in this thesis. The literature has been divided into four different categories, each of which discusses a number of papers. Additionally, supporting literature for the work discussed here can be found in \appref{relwork_extra}.



\section{Molecule parameterisation}
\seclab{simulations}
For running molecular simulations, a force-field model describing a molecule's interatomic interactions is required~\cite{canzar2012charge}. In the simulations, the force field requires a specific topology, which includes the molecule's atom types, bonds, bond angles and atomic charges. Using this information, a molecule can be divided into a set of non-overlapping charge groups: sets of \emph{connected} atoms each of whose charge is ideally equal to the molecule's total charge. The work of Canzar, El-Kebir, et al. shows the need for finding the atomic partial charges of a molecule. These are required for finding the charge groups, which in turn are essential for running molecule simulations.

Malde et al. present the Automated Force Field Topology Builder~(ATB)~\cite{malde2011automated}. The ATB is a web service that provides topologies which can be used in molecular simulations. It both acts as a repository for molecules that have already been parameterised, and is able to parameterise molecules itself. This automatic parameterisation involves quantum mechanical calculations, which are very complex and computationally intensive. As it is believed that there is no way to speed up these calculations, we developed a new approach. By comparing the charges found with the new approach to the charges that are present in the ATB topologies, this new approach can be validated.



\section{Interaction design}
\seclab{design}
One of the most important aspects of design is visibility~\cite{norman2002design}. Every interface should have visible features that can send the right messages to the user. It is very important that user actions do not have coincidental consequences. Otherwise, the user can develop wrong expectations of his actions, which may later result in problems while using the tool. What is also important, is that if a user \emph{does} make a mistake, he should be able to undo this. When this is \emph{not} possible, he may get easily frustrated and will stop using the tool. Additionally, the interface should not get in the way of the task that needs to be performed, leading to the user spending his time `using the interface', but should rather help him executing his tasks~\cite{norman1990interfaces}. Finally, the user should not get lost in an giant list of features. As many of these features will rarely be used, tools with less features, or where the advanced features are presented in a non-obtrusive way, are often the better ones.

When doing interaction design, one has to keep in mind that the appearance of a tool can have great impact on how well the user of that tool can carry out the tasks the tool is intended for~\cite{norman2002emotion}. Tools that are unattractive tend to focus the mind, which leads to better concentration. For tasks where something needs to be done quickly, this is good, but in cases where creative thinking is required, this will not provide a satisfying result. In that case, the thought processes should be broadened by having something that looks really good, thereby slightly distracting the user and allowing him to use creative thinking.

Not only the appearance of software tools is important, the form in which the information it comprises is presented can make a big difference as well. For certain types of data, it is highly beneficial to be represented in a visual form. This allows the viewers to perceive patterns and relationships that might be missed in tables and numbers~\cite{gallopoulos1994computer}. Providing users with the ability to modify view parameters at runtime can increase the benefits even further, as they then have the freedom to explore every aspect of the data. However, as discussed before, these options can cause application users to get lost in the list of options, no longer being able to execute the task they want to perform. Therefore, the options should generally be located in some initially closed menu.


\subsection[Principles]{Interaction design principles}
\seclab{id_principles}
Following the previously discussed and other research, a number of principles of interaction design has been identified.
Norman and Nielsen present the following principles of interaction design that are completely independent of technology in~\cite{norman2010gestural}:
\begin{itemize}[noitemsep,topsep=0pt,parsep=0pt,partopsep=0pt]
\item Visibility (also called perceived affordances or signifiers);
\item Feedback;
\item Consistency (also known as standards);
\item Non-destructive operations (hence the importance of undo);
\item Discoverability: all operations can be discovered by systematic exploration of menus;
\item Scalability: the operation should work on all screen sizes, small and large;
\item Reliability: operations should work and events should not happen randomly.
\end{itemize}


\subsection{Automation}
\seclab{id_automation}
Software and artificial intelligence developments allow for increasing possibilities in automating processes. The amount of interaction between humans and computers can be reduced, allowing humans to concentrate on other tasks~\cite{payne2000varying}. However, it depends on the situation when automation should be applied and to what extent. Automation reduces people's situational awareness, which may be undesirable in some environments. Furthermore, it may leave users feeling out of control when decisions are being made autonomously, rather than having a system that is only giving advise.

In order to determine what level of automation is suitable for a task, multiple interaction methods need to be designed. These often include a `naive' version that only does validation of user input, a `cooperative' version in which the user is given advise on what to do, and a mostly automatic `autonomous' version that only requires some initial parameterisation~\cite{payne2000varying,horvitz1999principles}. In a case study on route planning agents it was found that, for that purpose, a cooperative version delivered the best results~\cite{payne2000varying}. The users of the autonomous version complained they lacked the ability to control the system, while those who used the naive version complained the route planning process was tedious. However, this does not mean that the cooperative version is the silver bullet for man-machine interaction. In some cases, having full control might be necessary, or in other cases, where time consumption is really important, an autonomous version might be preferred.

Cooperative or `mixed-initiative' user interfaces have been studied in more detail by Horvitz~\cite{horvitz1999principles}. He has found that, in order for automation to work, automated activity should not occur before a user is ready for it. Delays in automation, on the other hand, can diminish its value, so it needs to be carefully timed. Among the most important factors for successful semi-automatic applications are: considering the computer's uncertainty about a user's goals, inferring the ideal action in light of costs, benefits and uncertainties, minimizing the cost of poor guesses, and providing mechanisms for efficient agent-user collaboration to refine results.

An example of a system that has implemented this cooperative interaction design is \verb|SALT|~\cite{marcus1987taking}, a tool for generating expert systems that are to be used in problem solving. Using \verb|SALT|, one will incrementally construct a design by either accepting or rejecting proposed design parameters, one parameter at a time. Constraints will constantly be checked, and, once a constraint violation is detected, the system will try to automatically remedy this. As it is extremely difficult to develop a system that has to perform a task for which expert knowledge is required, \verb|SALT| offers ways to backtrack the assigned parameters. It automatically detects parameters that might need to be modified, but require additional domain knowledge to be fixed properly.

At every step in \verb|SALT|'s incremental parameter assignment, the proposed values are provided in such order that the ones having the least negative effect come first. However, as this does not consider future assignments of other parameters, this means that the assignment may not always converge to a proper solution. This is where the previously discussed backtracking will be useful to see where the `wrong' decision has been made. In two field studies, \verb|SALT| has proven to be a good and useful tool for problem solving. However, it has also been found that the required degree of interaction can vary among different systems. Therefore, one should always study the context of a system before deciding on the level of automation that system will have.


\subsection{Abstraction}
\seclab{id_abstraction}
In order to be able to properly discuss an interaction design, this design needs to be represented at an abstract level. In~\cite{brehmer2013multi}, Brehmer and Munzner describe a multilevel typology for visualisation tasks. This typology consists of the answers to three questions: \textit{why} the task is performed, \textit{how} the task is performed and \textit{what} it pertains to. A list of predefined nodes is provided, using which those questions must be answered. By comparing multiple of these typologies, one is able to reason about the differences between two systems on an abstract level, which can be used to hypothesise what system will work better under which circumstances.



\section{Molecule software}
\seclab{software}
There is a lot of software available for displaying, drawing and editing molecules. These programs form an indispensable part of every molecular processing system~\cite{ertl2010molecular}. Throughout the years, molecule software has evolved from basic text editors, through clickable image maps, to full-on molecular structure sketching software. In the last few years, a new trend becomes visible. More and more cheminformatics applications are being brought to the web, allowing for many new ways of interaction. Furthermore, these new applications are being open sourced, allowing for new innovative variations or combinations of the old tools.

An example of a previously offline, closed source tool is \verb|JSmol|, originally known as just \verb|Jmol|~\cite{hanson2013jsmol}. This tool has been seamlessly transformed from a \verb|Java| applet to \verb|JavaScript|, without any visible visual difference. Furthermore, performance wise, there is only a minor difference between the two implementations. Another example is \verb|JSME|~\cite{bienfait2013jsme}, a tool that has been cross-compiled from its original \verb|Java| code to \verb|JavaScript| using the \verb|Google Web Toolkit| compiler. The transition to the web has resulted in the addition of several features, suggested by users from the new, bigger audience. What can be concluded from these two cases is that bringing molecule software to the web opens up a whole world of new possibilities, without having to give up on performance.

With the ongoing migration of molecule software to the web, new devices, such as smartphones and tablets, will be able to run the tools. As these devices promote different ways of user interaction and often have smaller screens, designs of the molecule software need to be reconsidered. \verb|TB Mobile| is a mobile app for identification of potential anti-tuberculosis molecules~\cite{ekins2013tb}. The app is structured in such way that there is always a small control bar at the top, and a large area for showing the content. Interaction with the app is handled by a small number of large buttons, allowing for easy touch controls. In order to work for different screen sizes, everything in the content area is scalable, and, in case not everything fits on the screen, scrollable. Usage of the app in practice has shown that it helps to improve the work flow of tuberculosis researchers, by lowering the barriers for accessing the information it provides.

The issue of showing a lot of data on a limited-size screen is also addressed by Ertl and Rohde~\cite{ertl2012molecule}. They took the concept of a word cloud, and transformed that to the so called Molecule Cloud. Here the highest scored molecules are the biggest and the lowest are the smallest, just like the highest and lowest scored words in a word cloud. None of the molecules overlap, and the sizes are divided such that the whole available screen space is filled. User studies have shown that these molecule clouds provide an easy way of finding the most relevant molecules. However, when many molecules need to be displayed in a small area, this means that either all molecules will be very small, or the lowest scored molecules need to be left out completely. Depending on the situation, this may not be desirable.


\subsection{Molecule visualisation}
\seclab{ms_visualisation}
An important component of molecule software is the visualisation of the molecules. Molecules can be represented in a textual format, but it is often beneficial to use a visual representation. This can help to illustrate various chemistry phenomena, such as the atomic structure. A study on Finnish upper secondary schools has shown that visual molecule software can be of great value~\cite{aksela2008computer}. The fact that students could interact with the molecules made it easier for them to quickly understand new chemical phenomena, and increased their interest in the subject.

In cases where detailed molecule information needs to be described in an abstract way, two-dimensional schematic views, as opposed to three-dimensional structure models, are usually the most useful~\cite{zhou2009molecular}. In a 2D view, one can easily view every atom of a molecule, without having to rotate it. Furthermore, it can easily show which atoms are connected and what type of bond connects them. This creates a good overview of the molecule, and allows for easy comparison of two molecules.

For digitally visualising a molecule, a textual representation of it is needed that can be interpreted by a computer and transformed to a form that includes positional data. There are many different textual molecule formats, including \verb|SMILES|~\cite{daylight1992daylight}, \verb|InChI|~\cite{heller2013inchi}, \verb|Mol2|~\cite{tripos2005tripos}, and \verb|PDB|~\cite{bernstein1977protein}. Each of these formats serves its own purpose. \verb|SMILES| is mainly concerned with expressing atom types and bonds, \verb|InChI| adds charge information, \verb|Mol2| contains information about 2D or 3D atom positions, and \verb|PDB| files contain a lot of additional information about the proteins they describe.

In order to find the positional data of a molecule, it is possible to implement a position calculation algorithm, or use an existing system for this. Alternatively, as data formats including the positional data also exist, it is also possible to obtain this data by converting the input file to a format that does include it. Several systems exist that do this, including \verb|ChemAxon Molconvert|~\cite{chemaxon2014molecule} and \verb|Open Babel|~\cite{oboyle2011open}. \verb|Molconvert| is a closed-source, commercial program that is part of the \verb|ChemAxon Marvin| package. It allows for conversion of most accepted formats, but can only be used in production after buying a license. \verb|Open Babel| on the other hand is an open-source, free to use chemical toolbox. It supports even more formats than \verb|Molconvert|, but its atom position calculations are slightly less accurate.

As a molecule can essentially be seen as a graph of bounded degree with labelled nodes and edges, certain aspects of graph visualisation can be used when drawing molecules~\cite{boissonnat2001structure}. For very large molecules, graph reduction algorithms may provide a way to fit the whole molecule on a limited size screen~\cite{batagelj2004pajek}. Nodes of that reduced molecule can then be expanded to reveal hidden parts of the molecule, and cut-outs of the whole molecule can be shown to see only its relevant parts. In some cases, however, it may be necessary for the application user to see every atom of the molecule at all times. It is therefore application dependent if, and to what extent graph visualisation aspects can be applied to molecule visualisation.


\subsection{General requirements}
\seclab{ms_requirements}
In the earlier discussed study on the use of molecule visualisation software in chemistry education in Finnish secondary education (see \secref{ms_visualisation}), and another study on the user of computer programs for biology teaching~\cite{taylor2013interface}, several requirements for molecule visualisation programs have been identified. Molecule software should be easy to use, visually appealing, able to save information, and moderately priced~\cite{aksela2008computer}. Furthermore, it should allow its users to rotate and scale molecules, and have logical ways of interaction. Additionally, it is highly beneficial to have different visual representations of the molecule, or to at least be able to modify the visualisation parameters at runtime. This allows the application user to fully tailor the visualisation to his needs. Finally, a clear and extensive manual is needed that not only describes the basic features of the system, but also the more advanced aspects in great detail.



\section{User studies}
\seclab{user_studies}
In Software Engineering, an often used method to evaluate a project is performing a user study. One type of user studies is the experiment, in which two (or more) different configurations of the examined tool are compared~\cite{wohlin2003empirical}. It is important to define exactly what one wants to validate, which should preferably be something that can easily be observed (e.g. `time required' or `number of clicks', but not `comprehensibility')~\cite{stein2009assessing}. Furthermore, it is important that all different configurations are semantically equivalent (i.e.\ doing the same thing), have an equal degree of compression (i.e.\ contain the same information) and are formatted into their cleanest and clearest extent.

Another type of user studies is the survey. In a survey, a large number of variables can easily be tested. One should be careful not to test too many, though, as this will make the analysis infeasible~\cite{wohlin2003empirical}. The questionnaire questions should follow some natural flow that embodies all aspects of a user's experience~\cite{tuch2013analyzing}. This encourages them to report their detailed experiences, and reduces the chances of subjects failing to report certain things simply because they forgot about them.

A variety of predefined surveys is available for usability testing, including \verb|SUS|~(System Usability Scale)~\cite{brooke2013sus} and \verb|UMUX|~(Usability Metric for User Experience)~\cite{lewis2013umux}. \verb|SUS| consists of ten items, each of which should be graded on a scale of 1 (strongly disagree) to 5 (strongly agree). From these items, a total score on a scale of 0-100 can be calculated, with the average system scoring 68 points~\cite{sauro2011measuring}. \verb|SUS| has been extensively tested and has shown to be very reliable, both in general and in comparison with other surveys. It has also been found to deliver correct results even when the number of test subjects is small, with a $90\%$ chance for a correct outcome with a group of only 10 people~\cite{tullis2004comparison}.

The \verb|SUS| questionnaire comes in two variants: one with alternating positive and negative statements, and one with only positive ones. The latter of these has been shown to have a higher reliability, presumably due to the effects of the alternating meaning of (dis)agreeing with a statement, which may confuse the user~\cite{lewis2013umux}. This finding has lead to the development of \verb|UMUX-LITE|, a questionnaire consisting of just the positive statements of \verb|UMUX|. \verb|UMUX-LITE| has been shown to yield results of equal reliability as those of the complete \verb|UMUX|, despite the fact that it consists of just two questions. Even though it is less reliable than \verb|SUS|, it is believed that when a system is subjected to both of these surveys, one can get very reliable results on the usability of that system.

On a final note about surveys, the fact that they are often straightforward makes them perfectly suitable for being held over the internet. It has been shown that the results of online surveys do not significantly differ from their offline counterparts~\cite{komarov2013crowdsourcing}. Nevertheless, one should always consider the effect of the lack of control of the environment on the result.
