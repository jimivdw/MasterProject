\chapter{Conclusion}
\chlab{conclusion}

In an attempt to speed up the molecule parameterisation process, we have developed a new approach, using which the charges of individual atoms in a molecule are retrieved from fragments of other molecules. We implemented this new approach in a system called the Online tool for Fragment-based Molecule Parameterisation~(\oframp). We designed two different versions of it, called consuming and producing, and compared them by doing a user study.

From the user studies, we have concluded that the producing interaction design allowed users to be done quicker, while the consuming version yielded more accurate results. The users showed a strong preference for the consuming design, as they liked the overview they had of the matching fragments, and felt like they were more in control than in the producing version. This has turned out to be an important aspect, and is required to make a good parameterisation of a molecule.

In addition to this, users rated the consuming version better than the producing one. With a \verb|SUS| score of 76, it is even rated above average, and good in general. As we consider user satisfaction and result correctness to be the most valuable aspects of the molecule parameterisation system, the consuming interaction design has been selected as the overall better of the two.

More in general, most of the common features of the two versions were appreciated as well. First, the graphical design of the application was found to be very clean and clear, and noted to look professional. Second, users were enthusiastic about the way in which the molecule was visualised, the ability to move it around and scale it, and the responsiveness of the visualisation. The selection details window was found to be very useful, and the demo mode helped them to quickly learn how to use the system. Finally, although not everyone used it, the help pages were highly appreciated for their extensive descriptions of every aspect of \oframp.

There were, of course, also some negative comments. Often, the found fragments were not ordered such that the actual best match ended up on top of the list. Furthermore, there were even situations in which barely matching fragments were rated to be very good. Although the development of the fragment finding system was not a part of this project, it needs to be improved further in order for \oframp{} to be able to attain its full potential.

Additionally, some suggestions have been provided on how the system can be improved further. They span from the addition of an auto-select mode for atoms, to the modification of certain atom colours. By implementing these suggested changes, we can further improve the quality of \oframp.

Once all alterations of the system have been made, we believe that it will become a tool using which a parameterisation can be obtained that is of comparable quality to one obtained using the conventional quantum-mechanical calculations. Furthermore, chemists will be able to fully parameterise a molecule using \oframp{} in less time than using the conventional methods.

We can therefore conclude that all of the hypotheses posed in \secref{ra_hypotheses} hold. The producing interaction design presented here turned out to take up less time than the consuming one. This version in turn, delivered more accurate results, was rated higher by its users and is therefore considered the overall better version. Finally, using \oframp, it should definitely be possible to create a molecule parameterisation of equal quality to the calculated ones, in less time.
