\chapter{Setting up \oframp}
\chlab{setup}

\lstset{language=sh, frame=single, basicstyle=\ttfamily\small, breaklines=true}

When one wants to set up the \oframp{} system, he should make sure that the requirements discussed in \secref{server_req} are met. Then, one first needs to set up \oapoc{} and \omfraf{}, after which \oframp{} can be properly configured. In this chapter, we provide detailed instructions on the complete process, following which the set up should be easy.


\vspace{-.2cm}
\section[\oapoc]{Setting up \oapoc}
\vspace{-.1cm}
When setting up \oapoc, first, \verb|Open Babel| needs to be downloaded\footnote{\url{http://openbabel.org/wiki/Get_Open_Babel}}. Make sure to install the right version for the used operating system, and follow the installation instructions that are provided on the download page. Once \verb|obabel| has been installed, the source code of \oapoc{} needs to be downloaded:
\begin{lstlisting}
$ git clone git@github.com:jimivdw/OAPoC.git <target>
\end{lstlisting}
Now, your web server (e.g. \verb|Apache|) needs to be configured such that \oapoc{} can be reached through the internet. More detailed instructions on how to configure a web server can be found online.

Everything should be up and running now. This can be validated by navigating with your browser to the configured root URL of your \oapoc{} instance, and submitting a simple MDS (e.g. `\texttt{CCC}'). If something fails, try to look at the configuration in \verb|<target>/atompos/atompos/main/util.py|, and see if anything should be changed there. This should not be the case, but might be necessary for some operating systems.



\vspace{-.1cm}
\section[\omfraf]{Setting up \omfraf}
\vspace{-.1cm}
Before one can set up \omfraf, the \verb|LEMON| graph library needs to be downloaded\footnote{\url{http://lemon.cs.elte.hu/trac/lemon/wiki/Downloads}} and installed following the installation instructions on the download page. Once this is done, \omfraf's source code should be downloaded:
\begin{lstlisting}
$ git clone git@github.com:jimivdw/OMFraF.git <target>
\end{lstlisting}
Next, the \verb|mop| system should be initialized. This can be done by executing the following commands:
\begin{lstlisting}
$ cd <target>
$ git submodule init
$ git submodule update
$ cd src/bin/mop/
\end{lstlisting}
Now, follow the compilation instructions present on the \verb|mop| project page\footnote{\url{https://github.com/melkebir/mop}}. Just like for \oapoc, configure your web server such that \omfraf{} can be reached through the internet.

You can now validate if everything works by navigating with your browser to the configured root URL of your \omfraf{} instance, and checking if repositories can be retrieved. Again, when something fails, one can try to look at the configuration in \verb|<target>/src/omfraf/main/util.py|, and see if anything should be changed there. This should generally not be the case, but might be necessary for some operating systems.



\section[\oframp]{Setting up \oframp}
Once both \oapoc{} and \omfraf{} have been set up, it is time to initialise \oframp. First, the source needs to be downloaded:
\begin{lstlisting}
$ git clone git@github.com:jimivdw/OFraMP.git <target>
\end{lstlisting}
Next, the URLs to \oapoc{} and \omfraf{} need to be specified in the configuration file. Open the file \verb|<target>/src/static/js/OFraMP/settings.js| in your preferred editor and look for the lines that look like this:
\begin{lstlisting}[language=JavaScript]
var DEFAULT_SETTINGS = {
  oapoc: {
    url: "http://vps955.directvps.nl/OAPoC/generate/",
    loadUrl: "http://vps955.directvps.nl/OAPoC/loadATB/",
    version: "1.0"
  },

  omfraf: {
    url: "http://vps955.directvps.nl/OMFraF/load/",
    repoUrl: "http://vps955.directvps.nl/OMFraF/repos/",
    generateUrl: "http://vps955.directvps.nl/OMFraF/generate/",
    version: "1.0"
  },
  // ...
};
\end{lstlisting}
Make sure to replace all URLs with those where your \oapoc{} and \omfraf{} servers are located. Now, one last time, configure your web server such that \oframp{} can be reached through the internet as well. Everything should be up and running, and \oframp{} will be ready for use.
