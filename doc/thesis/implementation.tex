\chapter{Implementation}
\chlab{implementation}

% Max. one figure at top of page
\setcounter{topnumber}{1}
\setcounter{bottomnumber}{1}
\setlength{\textfloatsep}{1em}
\renewcommand{\topfraction}{0.9}
\renewcommand{\bottomfraction}{0.9}
\renewcommand{\textfraction}{0.1}

A system for fragment-based molecule parameterisation has three distinct tasks: visualising a molecule, finding matching fragments for that molecule, and allowing for interaction with the molecule and its matching fragments. As these tasks can be performed in isolation, we have decided to implement the molecule parameterisation system as three separate tools.

The front-end of the system, where users will carry out the parameterisation process, is called the Online tool for Fragment-based Molecule Parameterisation~(\oframp). We also use this name to refer to the system as a whole, as it is the hart of the system and is dependent on the other systems. The molecule is visualised in \oframp\ as well, but the for this purpose essential calculations of atom positions are done in the Online tool for Atom Position Calculations~(\oapoc). Finally, finding matching fragments and sorting them based on relevance is done by the Online tool for Molecule Fragment Finding~(\omfraf).

The complete network diagram for the fragment-based molecule parameterisation system is shown in \figref{network_diagram}. As we can see here, \oframp{} is the central system, and the more computation-intensive tasks are carried out by \oapoc{} and \omfraf{}. In the remainder of this chapter, we will discuss each of the systems in more detail. Additionally, more details are provided in \appref{implementation_extra}.

\begin{sidewaysfigure}[p]
\center
\includegraphics[width=\textwidth]{img/network_diagram.pdf}
\vspace{1em}
\caption{Network diagram of \oframp{} and its supporting systems.}
\figlab{network_diagram}
\end{sidewaysfigure}



\section{Code statistics}
Following the current trend in web applications, \oframp{} has been implemented using the latest techniques from \verb|HTML5|, \verb|CSS3| and \verb|JavaScript|. As discussed in \secref{platform}, this allows for great portability and availability of the system. However, making use of the latest web technologies also comes at a cost. Older browsers generally lack support, and will therefore not be able to run the system. Luckily, according to the latest browser statistics from StatCounter~\cite{statcounter2014statcounter}, over 75\% of internet users worldwide uses a browser that has full support for these technologies. Thanks to many compatibility libraries in general, and the Internet Explorer \verb|canvas| fallback \verb|excanvas|~\cite{arvidsson2009explorercanvas} in particular, partial support can be provided for browsers used by an additional 20\% of people. A complete list of system requirements can be found in \secref{client_req}.

\clearpage
\lfootnotetext{loc_note}{This includes blank lines and lines containing just a closing bracket.}

Even though \oframp{} consists of only 3 pages~(the application itself and the help and download pages), it has become quite a big system; not only in functionality, but also code-wise. Combined with its 3 \verb|CSS| documents and 31 \verb|JavaScript| files, it consists of over 11,000 lines of code\lfootnoteref{loc_note}, or over 17,000 lines when the 10 used (and often modified) external libraries are counted as well. Detailed code statistics can be found in \tabref{code_statistics}.

\begin{table}
\begin{center}
\begin{tabular}{|l|c|c|c|c|}\hline
\textit{System} & \textit{\# Files} & \textit{\# LOC}\lfootnoteref{loc_note} & \textit{\# Files (incl. libs)} & \textit{\# LOC (incl. libs)}\lfootnoteref{loc_note} \\\hline
\oframp{} & 37 & 11,073 & 47 & 17,548 \\\hline
\oapoc{} & 8 & 578 & 15 & 926 \\\hline
\omfraf{} & 8 & 563 & 15 & 913 \\\hline\hline
\textit{Total} & 53 & 12,214 & 77 & 19,387 \\\hline
\end{tabular}
\caption{Code statistics of \oframp{} and its supporting systems.}
\tablab{code_statistics}
\end{center}
\end{table}

As \oapoc{} and \omfraf{} need to be available as web services, it has been decided to implement them using the \verb|Python| web framework \verb|Django|. Even though this creates a whole lot of unnecessary files and has a lot of unneeded functionality, it allows for a quick and easy setup of the systems. Furthermore, it provides the full functionality of \verb|Python|, which can be used to easily run external applications.

Size-wise, \oapoc{} and \omfraf{} are almost equally large. Both consist of 8 \verb|Python| files, containing just under 600 lines of code\lfootnoteref{loc_note} or, when the 7 additional \verb|Django| configuration files are included, around 900 lines. This brings the total number of manually written lines over 12,000, which will rise close to 20,000 when all automatically generated code and pre-existing libraries are included.



\section[\oframp]{The Online tool for Fragment-based Molecule Parameterisation}
\oframp{} is the central part of the fragment-based molecule parameterisation system. It contains the user interface and connects with \oapoc{} and \omfraf{}. As discussed in \secref{platform}, we have decided to implement the system as a web application, which the users can simply load in their web browser.


\subsection{Getting started}
\seclab{impl_starting}
Upon loading \oframp{} for the first time, one should see something similar to what is shown in \figref{impl_welcome}. It is explained what the tool is, and what it should be used for. Additionally, instructions are provided for first-time users and pointers are provided to the help pages. From this point, users can either start a short demonstration guide~(see \secref{impl_demo}) or submit a new molecule into the system, which we will discuss later.

\begin{figure}
\center
\includegraphics[width=.9\textwidth]{img/impl_welcome.png}
\caption{\oframp{} welcome screen.}
\figlab{impl_welcome}
\end{figure}

\Figref{impl_welcome} also shows the basic design of the system. We decided to go for a minimal look with a simple, purely textual logo. Besides the pastel colours of the logo, all text uses a dark shade of grey. Boxes and buttons all have a white background and a light grey, slightly rounded border.

Overly designed shiny elements are known to distract users from their tasks~\cite{norman1990interfaces}~(see also \secref{design}). In some cases this may be beneficial, but for the purpose of finding a good molecule parameterisation this is undesirable. Therefore, this minimal design tries to make sure the users stay focused on the task they need to perform, rather than on using the tool. Furthermore, the tool still looks good and decent, which should prevent the users from distrusting it due to poor design~\cite{norman2002emotion}~(see also \secref{design}).

\subsubsection{Submitting a molecule}
\seclab{impl_inserting}

\begin{figure}
\center
\includegraphics[width=.9\textwidth]{img/impl_inserting.png}
\vspace{-.2cm}
\caption{Insertion window for molecule data strings.}
\figlab{impl_inserting}
\vspace{-.2cm}
\end{figure}

In \figref{impl_inserting}, the insertion window for molecules is shown. This is where users should submit the molecules in a machine readable format, the so called molecule data string~(MDS). As discussed before, the system supports the \verb|PDB|, \verb|SMILES|, and \verb|InChI| formats, and the use of \verb|ATB ID|s~(see \secref{id_common}). When the MDS has been entered, the user can click the `Submit' button, after which the molecule will be displayed~(see \secref{impl_visualisation}). Additionally, the user can adjust the fragment repository and shell size that are to be used for finding matching fragments of other molecules~(see \secref{impl_omfraf}), or start the demonstration mode~(see \secref{impl_demo}).


\subsection{Visualisation}
\seclab{impl_visualisation}
Upon submitting the \verb|SMILES| data string `\verb|NCC(=O)CCO|' to \oframp, one should see something resembling what is shown in \figref{impl_visualising}. Similar to the molecule visualisation of the ATB~(see \figref{partial_charges} on \figpageref{partial_charges}), atoms are displayed as circles containing the atom type and with connecting lines that denote bonds. However, there are also a number of differences. First of all, charge group information is missing here, as these cannot be found before the atomic charges are available. Second, obviously, the atom charges are missing, but they will be added to the visualisation as soon as a matching fragment is selected~(see \secref{impl_parameterisation}). Third, as can be seen at the bottom-most $O$ atom, double bonds are visualised properly as a double line. The same is true for triple bonds~(triple line), and aromatic bonds~(dotted line on the inside of the aromatic cycle they are a part of).

\begin{figure}
\center
\includegraphics[width=.9\textwidth]{img/impl_visualising.png}
\caption{\oframp{} visualisation of \texttt{NCC(=O)CCO}.}
\figlab{impl_visualising}
\end{figure}

Finally, what is probably the most notable difference, is the fact that the $H$ atoms are grouped with their base atoms\footnote{This, along with many other settings, can be altered. See \secref{impl_settings} for more information.}. Thanks to this, they do not need to be drawn separately, leaving more space for the other atoms. This enhances the overview of the complete molecule. Furthermore, in chemistry, it is quite common to have this combination, as these groups can be treated as a single atom in many calculations.

\subsubsection{Interaction}
At this point, the molecule can already be interacted with. It can be moved around by holding down the \emph{left} mouse button and dragging the mouse. This is mostly useful for larger molecules that may not completely fit on smaller screens. To provide further support for large molecules, the user has the opportunity to zoom in on certain sections of the molecule, or zoom out to make more atoms fit on the screen. Zooming can be done using the mouse wheel, or the zoom buttons located in the advanced controls menu. This menu can be found by clicking the button with the down pointing arrow~(see \figref{impl_visualising}).


\subsection{Parameterisation}
\seclab{impl_parameterisation}
As soon as the matching fragments for the molecule have been generated, the `Loading fragments...' message as visible in \figref{impl_visualising} will go away. It is now possible to start the fragment-based parameterisation process. We will discuss the two implemented behaviours in the next sections.

\subsubsection{\IDA\ version}
In order to start the parameterisation process in the \IDa\ version of \oframp, a selection needs to be made. This selection can consist of a single atom or multiple \emph{connected} atoms, and should contain the atoms for which the user wants to find matching fragments. In order to make a single-atom selection, the user simply needs to click on that atom~(see \figref{select_1}). There are multiple ways to create multi-atom selections, the easiest of which is holding down the \emph{right} mouse button and dragging the mouse to create a selection rectangle as seen in \figref{select_2}. Alternatively, one could hold down the \verb|Ctrl| key while clicking atoms, or activate the selection modification mode, as seen in the selection details window in \figref{find_1}.

\begin{figure}
\centering
\begin{subfigure}[t]{0.45\textwidth}
\centering
\includegraphics[width=\textwidth]{img/select_1.png}
\caption{Making a single-atom selection.}
\figlab{select_1}
\end{subfigure}%
\qquad
\begin{subfigure}[t]{0.45\textwidth}
\centering
\includegraphics[width=\textwidth]{img/select_2.png}
\caption{Making a multi-atom selection.}
\figlab{select_2}
\end{subfigure}
\caption{\oframp{} atom selection.}
\figlab{selecting}
\end{figure}

\begin{figure}
\center
\includegraphics[width=.9\textwidth]{img/find_1.png}
\caption{Found fragments.}
\figlab{find_1}
\end{figure}

When a selection has been made, matching fragments will automatically be retrieved from \omfraf~(see \secref{impl_omfraf}). These fragments will be shown in a sidebar on the right-hand side of the screen, and atom selection details will appear on the left-hand side~(see \figref{find_1}). Upon clicking a fragment, its charges will be previewed on the molecule~(see \figref{find_2}). The users can then decide if they want to apply the charges of this fragment by clicking the `Select fragment' button that appears on the fragment, preview a different fragment by selecting another one, or inspect the fragment in the context of the molecule it originates from by clicking the `Show molecule' button.

\begin{figure}
\center
\includegraphics[width=.9\textwidth]{img/find_2.png}
\caption{Previewing the second found fragment.}
\figlab{find_2}
\end{figure}

As soon as the users click the `Select fragment' button, the charges from that fragment will be copied to the molecule. The selection will be cleared, and, because of that, the list of found fragments will be closed. The users can now select another atom or set of atoms they want to parameterise, in order to complete the parameterisation. This will be discussed in more detail in \secref{impl_completing}.

\subsubsection{\IDB\ version}

\begin{figure}
\center
\includegraphics[width=.9\textwidth]{img/find_3.png}
\caption{Ready to start parameterisation.}
\figlab{find_3}
\end{figure}

In the \IDb\ version of \oframp, the parameterisation process can be started by clicking the `Start parameterising' button shown in \figref{find_3}. The system will now automatically select a random atom at the edge of the molecule, for which fragments will be retrieved. The molecule will be centred on that atom, and the charges of the highest rated fragment will be previewed~(see \figref{find_4}). In that figure, the atoms with the green background and the thicker border are the atoms occurring in the found fragment.

\begin{figure}
\center
\includegraphics[width=.9\textwidth]{img/find_4.png}
\caption{Previewing a found fragment.}
\figlab{find_4}
\end{figure}

The users now need to decide whether they think this proposed fragment is a good match with the molecule. In order to help them make this decision, they can view the fragment in the context of the molecule it originates from by clicking the `View original' button~(see \figref{find_4}). When they reject the fragment, the next best rated fragment will be previewed, up until the point where there are no found fragments left. Either the final fragment needs to be accepted then, or the users can undo some of their rejections using the `Undo' button at the top of the page.

Upon acceptance of a fragment, the charges of that fragment are copied to the molecule. The system will then find another \emph{unparameterised} atom for which fragments will be proposed. This atom is preferably connected to the last accepted fragment, or, when no unparameterised atoms are neighbouring that fragment, positioned on the edge of the molecule. The user can now continue to parameterise the molecule, which will be discussed in more detail in \secref{impl_completing}.


\subsection{Completing parameterisation}
\seclab{impl_completing}

\begin{figure}
\center
\includegraphics[height=140px]{img/conflict.png}
\caption{Molecule with a charge conflict.}
\figlab{conflict}
\end{figure}

During the course of parameterising a molecule, it is possible to encounter fragments that contain atoms to which a charge has already been assigned. These atoms are given a yellow background, as can be seen in \figref{conflict}. When such fragment is selected by the user, it is dependent on which \oframp{} version is used what will happen. As discussed in \secref{id_smart}, the \IDb\ version will automatically assign the average of the two charges. In the \IDa\ version of the tool, a pop-up will be shown, asking the user to pick a solution method for solving the charge conflict.

In some cases, it can happen that no matching fragments are found for an atom. This will be indicated by colouring that atom with a pink/red background, as can be seen for the rightmost $CH_{2}$ group in \figref{conflict}~(and also \figref{selecting, find_1, find_2, find_3, find_4}). Upon selecting these atoms, no fragments will be returned. It is possible to parameterise them manually, in the same way one can fine-tune atom charges. This will be discussed later.

\begin{figure}
\center
\includegraphics[width=.9\textwidth]{img/finished.png}
\caption{Notification of parameterisation completion.}
\figlab{finished}
\end{figure}

When a charge has been assigned to all \emph{parameterisable} atoms~(i.e.\ atoms for which fragments can be found), the users will be notified of this with a pop-up similar to the one shown in \figref{finished}. As it says there, they can now fine-tune the atom charges~(see below), download the resulting parameterisation, or enter a new molecule.

\subsubsection{Fine-tuning charges}
\seclab{impl_finetuning}
Charges of fragments of molecules will often not precisely match the charges of another molecule. Therefore, the users have the ability to manually adjust atom charges when they wish to do so. In order to do this, they needs to select the atoms they wish to adjust the charges for. In the selection details panel, the `Show' button in the `Selected atoms' row should be clicked. A table containing detailed information about the atom will be shown, as can be seen in \figref{editing}.

\begin{figure}
\center
\includegraphics[width=.9\textwidth]{img/editing.png}
\caption{Manually fine-tuning or adding charges.}
\figlab{editing}
\end{figure}

The atom details table contains information about all atoms that are part of the selection. Hovering with the mouse over a row will highlight the corresponding atom in the molecule, which can be seen for the $O$ atom in \figref{editing}. The `Elem' column contains the atom element and its \verb|IACM atom type|~(in parentheses). The charge is shown, which can be edited by clicking the `Edit' button in the column next to it. Finally, clicking the `Show' button in the `Frags' column, will show the fragments that were used to get the current atom charge, as discussed in \secref{id_common}.


\subsection{Other features}
Besides the features required for the normal use of \oframp{} described in the previous sections, some other features have been developed to further improve the user experience. The following sections will discuss the Demo mode, modification of visualisation parameters, and the \oframp{} Help pages.

\subsubsection{Demo mode}
\seclab{impl_demo}
In order to help new users learn to use the system, we have developed a Demo mode. Both implementations of \oframp{} have their own Demo implementation, suited for that version. It can be started by clicking the `Start demo' button, as discussed in \secref{impl_starting}.

\begin{figure}
\center
\includegraphics[width=.9\textwidth]{img/demo_1.png}
\caption{Start of both Demo modes.}
\figlab{demo_1}
\end{figure}

The Demo mode will guide the users through the basic steps that are required to completely parameterise a molecule. They will receive detailed instructions on what buttons to press at which time, and other buttons will be disabled. Upon starting the demo mode, one should see something similar to what is shown in \figref{demo_1}. A simple molecule data string will automatically be entered into the input field, to ensure that this is a molecule that can be used to easily explain how to use the system.

\begin{figure}
\center
\includegraphics[width=.9\textwidth]{img/demo_2.png}
\caption{Parameterisation in the Demo mode of the \IDb\ version of \oframp{}.}
\figlab{demo_2}
\end{figure}

The user will be guided through most aspects of \oframp, including interaction with the molecule visualisation, fragment selection and comparison~(see \figref{demo_2}), and retrieving the results. By allowing for user interaction with the demonstration, we hope that the Demo mode will not only serve as an example, but will also help the user to learn how to use \oframp{} in a different case. This way, it should have both the benefits of learning by example and learning by familiarisation~\cite{sweller1994cognitive}~(see also \secref{id_learning}).

\subsubsection{Modifying visualisation parameters}
\seclab{impl_settings}
Research on both data visualisation in general~\cite{gallopoulos1994computer}~(see also \secref{design}) and molecule visualisation in particular~\cite{aksela2008computer,taylor2013interface}~(see also \secref{ms_requirements}) has shown that being able to modify visualisation parameters at runtime will greatly enhance the user experience. Additionally, it will help users to carry out the task they intend to complete. Therefore, most of the molecule visualisation parameters, such as the atom radius or grouping of $H$ atoms, can dynamically be modified. This is not included in the demo, but information about it can be found on the \oframp{} Help pages, which we will discuss in the next section.

\subsubsection{Help}
\seclab{impl_help}
Despite the fact that using \oframp{} should be intuitive, and the presence of its Demo mode, some users prefer textual instructions. Furthermore, some of the advanced features that are not covered by the demonstration need to be documented as well. Therefore, we have created extensive Help pages for both versions of the system~(see \figref{help}). It includes a little introduction on \oframp, basic instructions on how to parameterise a molecule, detailed descriptions of the advanced features, and instructions on how to solve common problems.

\begin{figure}
\center
\includegraphics[width=.9\textwidth]{img/help.png}
\caption{\oframp{} Help for the \IDa\ version: selection instructions.}
\figlab{help}
\end{figure}

As can be seen in \figref{help}, the Help page uses the same type of design as \oframp, consisting of white boxes with grey, slightly rounded borders. Once again, this simplicity makes sure users are not distracted by the interface, but will be able to find the information they need. Furthermore, the consistency in the layout follows the interaction design principles~\cite{norman2010gestural,blair2008user}~(see also \secref{id_principles}) to enhance user satisfaction.



\section[\oapoc]{The Online tool for Atom Position Calculations}
\seclab{impl_oapoc}
The atom positions and bonds that are required for visualising molecules are found by \oapoc. As can be seen in \figref{network_diagram} on \figpageref{network_diagram}, \oapoc{} can calculate the molecule data either from an \verb|ATB ID|, or a molecule data string in an optionally specified format\footnote{When not provided, the format will be inferred automatically, which may be error-prone.}. Internally, this input will be converted to some other format that includes the atom position and bond information, and can easily be parsed and interpreted.

For converting between different molecule data formats, we have decided to go with \verb|Open Babel|~\cite{oboyle2011open}~(see also \secref{ms_visualisation}). This tool can be used easily and for free, and has support for most common MDS formats. It has shown to yield good atom position data, and performs quite well.



\section[\omfraf]{The Online tool for Molecule Fragment Finding}
\seclab{impl_omfraf}
Perhaps the most important part of the fragment-based molecule parameterisation system is the part that finds matching fragments for the input molecule. In our implementation, this is done by \omfraf. Provided with the input molecule, a repository of molecules that can be queried for fragments, and a shell size within which the atoms should match, the tool can generate a list of matching fragments.

As one needs to have knowledge about some complex chemical concepts to be able to determine whether a similar fragment is a match or not, we needed to use an external tool for this. In \omfraf, fragments are generated by El-Kebir's \verb|fragments| tool, from the \verb|mop| project~\cite{elkebir2014molecule}.

Provided with a target molecule file and another input molecule file, the \verb|fragments| tool will find all maximal fragments in the input molecule file that match with parts of the target molecule. It will score the fragments based on, among others, their size, such that they can be presented in order of relevance. In order to be a match, not only the atoms that are part of the fragment must match; the atoms in a so-called shell around the fragment should also match.

\clearpage

\begin{figure}
\centering
\begin{subfigure}[t]{0.29\textwidth}
\centering
\includegraphics[width=\textwidth]{img/shell_1.png}
\caption{Target atoms that should occur in any fragment, highlighted in blue.}
\figlab{shell_1}
\end{subfigure}%
\qquad
\begin{subfigure}[t]{0.29\textwidth}
\centering
\includegraphics[width=\textwidth]{img/shell_2.png}
\caption{Matching fragment, highlighted in green.}
\figlab{shell_2}
\end{subfigure}%
\qquad
\begin{subfigure}[t]{0.29\textwidth}
\centering
\includegraphics[width=\textwidth]{img/shell_3.png}
\caption{Non-matching fragment, highlighted in red / pink.}
\figlab{shell_3}
\end{subfigure}
\caption{Fragment matching with shell size $1$.}
\figlab{shell_matching}
\end{figure}

\Figref{shell_matching} shows the concepts of fragment matching for a shell size of $1$. As we can see, in \figref{shell_2}, the neighbouring atoms of the selected $CH_{2}$ group are $C$ and $N$ atoms, just like in \figref{shell_1}. In \figref{shell_3}, the $C$ atom to the right still matches, but the atom on the left-hand side no longer matches, as this is a $C$ atom. Should the shell size be increased further, to $2$, for example, then the second fragment would no longer be matching, due to the $CH_{3}$ group connected to the $NH$ group, and the $CH_{2}$ group connected to the $C$ atom to the right. The former is not present in the original molecule at all, the latter should be an $O$ atom to match.

Once the fragments have been generated, the user of \oframp{} is able to start finding fragments. For this purpose, the generated list of fragments will be queried to retrieve the list of fragments in which a certain atom or set of atoms~(the selection) occurs.
