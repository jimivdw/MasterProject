\chapter{Research approach}
\chlab{approach}

In this project, as discussed before, we design and implement a tool for fragment-based molecule parameterisation, in order to answer the research questions that were discussed in \chref{introduction}. In this chapter, we discuss the taken approach. 

Since the idea of fragment-based molecule parameterisation is a new concept, it comes as no surprise that there are no existing tools for this. Furthermore, no tools exist that allow for easy comparison of molecules, or molecule fragments, either. This means that there is no baseline to which the developed tool can be compared. In order to be able to say something about the quality of the tool, we will have to make two different implementations of it~(see \secref{ra_versions}).

In order to evaluate the two implementations, both of them will be subjected to a user study~(see \secref{ra_studies}). This will help to determine if the system works properly and whether it is fit for its tasks. Furthermore, we will be able to answer the project's research questions, and test the hypotheses that will be discussed in \secref{ra_hypotheses}.


\section{Two versions}
\seclab{ra_versions}
There are a few axes along which the differences between the two implementations can be made. One could, for instance, compare two tools with varying visualisation methods. Visualising molecules, however, is a well-exhausted field of research, leaving very little room for new ideas.

Another possible variable in the tool is the way its users will interact with it and especially its degree of automation. Varying this degree has been the subject of several studies~(e.g.~\cite{payne2000varying, horvitz1999principles, marcus1987taking, norman1990problem}, see also \secref{id_automation}), all of which concluded the degree to which automation can be applied is highly dependent on the context of the system and sometimes even to the situation in which the system is used.

As discussed in \secref{id_automation}, usually three levels of automation are implemented, creating a naive, cooperative and autonomous version of the system~\cite{payne2000varying}. As it is considered hard to parameterise a molecule based on fragments of other molecules, and little is known on what is the best way of doing this, an autonomous version of a tool that does this cannot yet be developed. The other two versions, however, seem to be perfectly implementable and both can be considered useful.

It has therefore been decided to implement a naive version of the system that has barely any automation at all, and a cooperative version that will continuously make suggestions to the user. They have been called the consuming and producing version respectively, to reflect the differences in their interaction methods~(see \secref{id_versions}). In \chref{design}, the interaction designs for these two versions are discussed in detail, and their implementations are covered in \chref{implementation}.



\section{User studies}
\seclab{ra_studies}
As mentioned before, both versions of the developed system will be subjected to a number of user studies. Unfortunately, we cannot just ask anyone partake in this experiment, as it is a system that is aimed to be used by experienced chemists who understand the details of molecule parameterisation. Luckily, project supervisors Klau and El-Kebir keep a close connection with a substantial group of chemistry researchers, creating the opportunity of asking them to participate in a user study. They were interested in the concept of fragment-based molecule parameterisation, and willing to test a system that does that.

In the user studies, the participants will be split up in two groups, where each person is randomly assigned to a group and both groups are of equal size. This way, it is made sure that both groups are representative, which is essential for user studies~\cite{wohlin2003empirical}~(see also \secref{user_studies}). The groups will take part in an experiment that compares the two versions of the molecule parameterisation tool. Due to the limited number of test subjects, every participant will be asked to test both the consuming and the producing version of the system. As the effects of the order in which two systems are evaluated are often remarkable, one group will start the evaluation with the consuming version, followed by the producing version, while the other group will do this the other way around.

For both versions of the tool, the experiment participants will be instructed to parameterise a couple of molecules. After this, they will be asked to fill in a short questionnaire containing questions about that specific version of the system. When they have tested and answered questions about both versions of the tool, users will be asked to submit one final questionnaire. This one will contain more general questions about having a system for molecule parameterisation, and ask for their version preference.

Due to the large number of tasks the experiment participants have to complete, we have decided to have them parameterise only 2 molecules per version of the system, i.e.\ 4 molecules in total. With an estimated time requirement of 10 minutes per questionnaire, participants will already be spending half an hour on that. As the experiment should not take up more than an hour of the participants' time, this leaves only 30 minutes for the actual molecule parameterisation. When loading times are considered, we must conclude that there is only time for 4 molecules in total.

Even though testing more than two molecules per version of the system would be better, it is not necessarily a problem to only test two. Many problems with a system will be found in the first two test runs, after which the number of newly discovered problems decreases rapidly~\cite{krug2006dont, nielsen2000you}. Despite the fact that some problems may be left undiscovered, the most common flaws will usually be spotted in the first few runs.

As many of the available participants are scattered around the world, and in order to save time, it has been decided to conduct the experiment over the internet. The potential application users will be contacted via email, and use their own machines to run the system. Additionally, they will submit the answers to the questionnaires using some online form.

In order to be able to obtain usage statistics from the user studies, an extensive logging mechanism will have to be built into the system. Later, this can be used to precisely reproduce the user's actions. This will mediate the loss of information due to the fact that the user studies are performed over the internet, with no human observing the participants. If any odd results are found, the logs can potentially help to pinpoint what went wrong, and, more importantly, it can be used to obtain all kinds of statistics about the molecule parameterisation system.



\section{Hypotheses}
\seclab{ra_hypotheses}
According to Jonassen's dichotomy of problems~\cite{jonassen2000toward}~(see also \secref{id_learning}), fragment-based molecule parameterisation is an ill-structured problem. This means that the problem is quite complex and is therefore hard to solve. Due to this, users need to be ensured that there is a way to find a solution, or they will simply give up. This should be the case for both versions of the system, but is better facilitated by the consuming version than the producing one. In the consuming version, users are not relying on a system to automatically come up with suggestions, but rather do everything themselves. They will know that, by simply repeating the same process a few times, they will get a result. The fact that they have everything in their own hands will make them feel more comfortable about getting to a good solution.

In the earlier mentioned study of Payne, Sycara, and Lewis~\cite{payne2000varying}, it was found that the cooperative version of the examined system had great benefits in time consumption and user guidance, where the naive version excelled in user freedom. In the general, straight-forward cases studied there, the cooperative version delivered the best results, but, in more complex situations, it was no match for the naive version. As fragment-based molecule parameterisation is an ill-structured, and therefore complex problem, we believe that the following hypotheses will hold:
\begin{quote}
The producing version of the molecule parameterisation system will take less time to use than the consuming version.
\end{quote}
\begin{quote}
The consuming version of the molecule parameterisation system will yield better results than the producing version.
\end{quote}

When the participants of the study on automating RPAs~\cite{payne2000varying} were asked what they thought about the different versions of the system, they noted that using the consuming version was a tedious process that cost a lot of time. However, in some more complex cases, they found it really important to have full control of the system. As mentioned before, parameterising a molecule based on fragments of other molecules is a complex task in itself. Therefore, as we consider user satisfaction and result correctness to be the most important aspects in this case, we formulate the following hypothesis about the system in general:
\begin{quote}
For a tool used for fragment-based molecule parameterisation, it is better for users to have full, manual control, than to automate certain parts.
\end{quote}

On a more general level, both versions of the interaction design rely on the idea that fragment-based molecule parameterisation should be chemically correct. As this has not been proven yet, we aim to determine if it is possible in this project. Both versions of the system will be designed in such way that the user should be able to quickly parameterise a molecule. Furthermore, as it has been proven that properly matching parts of molecules have the same atomic charges, the second, for both versions of the system essential hypothesis, is as follows:
\begin{quote}
It is possible to parameterise a molecule based on fragments of other molecules; both correctly and in a reasonable amount of time.
\end{quote}

When this hypothesis does not hold, fragment-based molecule parameterisation will show not to be a good concept after all. This will also render the comparison of the two interaction designs invalid, as both are then designed for a faulty task, and have been evaluated using that. We consider it very unlikely that this would happen, since the similarity has been proven in theory, and the parameterisation times cannot get much worse than those of the quantum-mechanical calculations.
