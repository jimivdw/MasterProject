\abstract{
The current process of calculating atom charges uses complex-quantum mechanical calculations that take a long time to complete. A new approach has been proposes, where unknown atom charges are found using fragments from similar molecules for which the charge is known. This approach has been implemented in \oframp, a system that allows chemists to manually do this charge assignment. Two different interaction designs for it have been made and implemented, and were compared in a user study. It turned out that a `naive' interaction design, with only minimal automation, is the best for the task at hand, as this gave users more control and provided a better overview. Additionally, it was found that this manual process can be completed relatively fast, and, with a few improvements to the fragment matching algorithm, deliver good results.

}

\begin{comment}
When molecules need to be simulated, the individual charges of the atoms that molecule consists of need to be known. Currently, these charges are found using complex quantum-mechanical calculations that are computationally intensive and take long times to complete. There are ideas for a different approach, where unknown atom charges are found using fragments from similar molecules for which the charge is known.
\\[.5em]
A system, called \oframp, has been developed that allows chemists to manually do this charge assignment. Two different interaction designs for this have been made and implemented, and were compared in a user study. It turned out that a `naive' interaction design, with only minimal automation, is the best for the task at hand, as this gave users more control and provided a better overview. Additionally, it was found that this manual process can be completed relatively fast, and, with a few improvements to the fragment matching algorithm, deliver good results.
\end{comment}
