\abstract{
Computational chemists frequently use molecular simulations in computer-aided drug design. As input, these simulations require molecules that have been parameterised with atom charges. Typically, these charges are obtained from complex quantum-mechanical calculations, which are only feasible for small molecules. We propose a new approach for molecule parameterisation, which uses matching fragments from a repository of molecules that have already been parameterised. This approach has been implemented in a system called \oframp, for which two different interaction designs have been compared in a user study. We found that a \IDa\ interaction design without automation yields better results than one that proactively makes suggestions to the user. This version is also preferred by the experiment participants. With a few improvements to the fragment matching algorithm, \oframp\ should become a system that produces high-quality results in a shorter time than the current quantum-mechanical calculation systems.
}
