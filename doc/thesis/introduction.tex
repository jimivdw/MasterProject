\chapter{Introduction}
\chlab{introduction}

In the world of biomolecular science, molecular simulations are becoming increasingly important. These simulations are used in many different fields, among which are computational chemistry and drug design. For these simulations, many different parameters are needed, including detailed information about the molecules that will be simulated. One important parameter here is the list of atomic partial charges of the molecule~(see \chref{analysis}). Calculating these charges, however, is currently only feasible for small molecules, and can take a very long time.

It is believed that, using the current approach, the completion time of the calculation process can only be reduced by improving computer hardware. Therefore, in order to speed up the process of finding atomic partial charges, a different approach needs to be developed.

Similar parts of different molecules often have roughly the same atomic charges. Therefore, it should be possible to assign the charges of a molecule based on the known charges of parts of other molecules. We expect that experienced chemists will be able to find the best available matching molecule fragment out of a set of possible ones. Therefore, we design a system that allows for parameterisation of a molecule using a set of fragments of other molecules.

Given the short problem description above, we formulate the main research question for this project as follows:
\begin{quote}
How should chemists interact with a tool for fragment-based molecule parameterisation, such that using this tool yields results of comparable quality to using conventional methods, while taking less time to complete?
\end{quote}

This question can be split up into the following sub-questions:
\begin{enumerate}
\item How should chemists interact with a system for finding atomic partial charges using similar fragments of other molecules?
\item Compared to the computed charges, how accurate are the charges obtained using manual charge assignment?
\item Is it feasible to do manual partial charge assignment in a reasonable amount of time, i.e.\ faster than performing computations?
\end{enumerate}

In order to answer these questions, we have designed and implemented two slightly different systems for fragment-based molecule parameterisation. These two versions have been compared in a user study, where participants were asked to parameterise a few molecules using the system. Using the outcomes of the user study, we can determine which of the interaction designs is the best fit for the task of fragment-based molecule parameterisation, and if this task makes any chemical sense at all.

First of all, we found that it \emph{is} possible to parameterise a molecule based on fragments of other molecules, and get a result comparable to that obtained using the conventional quantum-mechanical calculations. Second, it turns out that a system that makes suggestions to the user takes up less time, where one that only reacts to user input yields better results.

In addition to using the system, user studies participants were asked to fill in a few questionnaires. An analysis of their answers showed that system users had a strong preference for the version that did not make any suggestions, but instead provided them with full control over the parameterisation process. Using the additional comments they made, we can further improve the design for that version of the system, in order to make it a system that can be of great value in biomolecular research.



\section{Outline}
In the remainder of this document, we will discuss the designed system and obtained results in detail. First of all, in \chref{analysis}, we provide a more detailed analysis of the molecule parameterisation problem. We discuss related work in \chref{relwork}, and discuss the research approach discussed in \chref{approach}. The two interaction designs can be found in \chref{design}, followed by implementation details for the whole system in \chref{implementation}. We discuss the evaluation methods for the project in \chref{evaluation}, followed by the experiment results in \chref{results}. In \chref{discussion} we provide a discussion of the results, along with threats to their validity and future work. Finally, we present the conclusions for the project in \chref{conclusion}.
