\chapter{Introduction}
\chlab{introduction}

Blah blah intro

\nlipsum

\section{Outline}
Outline of the document

\nlipsum

\begin{comment}
In the world of biomolecular science, molecular simulations are gaining in importance. In order to run these simulations, a set of parameters is needed, including the atomic partial charges of a molecule. Calculating these charges, however, can take a lot of time, even on advanced computation clusters.

As it is seemingly impossible to speed up the partial charge calculations, an alternative way to retrieve them should be developed. It is known that similar fragments in two different molecules often have roughly the same atomic charges. Therefore, it should be possible to assign the charges of a molecule based on the known charges of fragments of other molecules. It is expected that humans (read: experienced scientists) will be able to find the best matches out of a set of possible ones. Therefore, a tool needs to be designed that allows them to parameterise a molecule based on a set of related fragments of other molecules.

In this research project, such tool will be designed. Two slightly different prototypes of the tool will be implemented and compared to see which of these has the best interaction design for the task at hand. Furthermore, it will be evaluated whether such tool can truly improve the parameterisation process of molecules. Overall, the project aims to answer the following research questions:
\begin{itemize}
\item What is the best way for chemists to interact with a tool for assigning atomic charges based on those of similar fragments in other molecules?
\item Is it feasible to do manual partial charge assignment in a reasonable amount of time?
\item Compared to computed partial charges, how well can manual atomic charge assignment be done?
\end{itemize}
\end{comment}