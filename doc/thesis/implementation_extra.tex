\chapter[Additional impl.]{Additional implementation details}
\chlab{implementation_extra}


\section[\oframp]{The Online tool for Fragment-based Molecule Parameterisation}
\subsection{Visualisation}
Once a molecule data string has been entered into the system, the positions of its atoms and the connections between them are retrieved from \oapoc~(see \secref{impl_oapoc}). This data is then transformed to a \verb|JavaScript| \verb|Molecule| instance, as can be seen in \figref{oframp_class}. \verb|Molecule| instances can be visualised using a \verb|MoleculeViewer|, which will draw the molecule onto an \verb|HTML| \verb|canvas|.

\begin{figure}
\center
\includegraphics[width=\textwidth]{img/oframp_class.pdf}
\caption{Simplified class diagram of \oframp.}
\figlab{oframp_class}
\end{figure}

%

\subsubsection{Deoverlapping}
As discussed in \secref{ms_visualisation} and~\cite{clark2006structure}, overlap in atoms and bonds still occurs in modern molecule visualisation software. In this case, where the atoms have a relatively large radius to make room for displaying the charge, chances are even higher. To temper the effects of this, a simple deoverlap algorithm has been implemented. \Figref{impl_deoverlap} shows the three different types of deoverlapping that \oframp{} can do. Here, the light blue line indicates a vector that is used to determine if there is any overlap, the red line indicates the overlapping part, and the green arrow~(of equal length to the red line) shows where the atom centre should be moved to in order to resolve the overlap.

\begin{figure}
\center
\includegraphics[width=\textwidth]{img/deoverlap.pdf}
\caption{Concepts of deoverlap.}
\figlab{impl_deoverlap}
\end{figure}

By default, for performance reasons, only the deoverlapping of atoms is enabled. This is the easiest to detect and solve, but also the most important one. Atoms that overlap almost entirely might not be spotted by the user, and atom types and charges may become hidden due to another atom lying on top of them. As can be seen in \figref{impl_deoverlap}, this type of deoverlap works by calculating the distance between the two atom centres and subtracting the atom radius from that twice~(once for each atom). If this distance is smaller than $0$, both atom centres will be moved along the line that connects the two for a distance that is equal to the overlap distance. This ensures that the atoms will no longer overlap, and even creates some room between them.

In order to solve atoms overlapping bonds, the perpendicular distance from the atom centre to the bond needs to be calculated. When this distance is smaller than the atom radius, the atom needs to be moved along the perpendicular for a distance that is equal to the overlap distance. This will put the atom right next to the bond.

The third and final type of deoverlapping, the so-called decrossing of bonds, is also the hardest one. First, it needs to be determined if the two bonds intersect each other. When that is the case, the intersection point needs to be found. The overlap distance here is equal to the distance from the atom centre to the intersection point, plus the radius of the atom. The atom then needs to be moved along its bond for a distance that is equal to the overlap distance, in order to solve the crossing bonds.

Unfortunately, solving one occurrence of overlap could result in overlap occurring in a different place, as moving an atom could potentially move it on top of another. Therefore, the deoverlap process will need to repeat itself to solve all overlap in the molecule. However, it is possible that the solution of one overlap occurrence is the exact opposite of another, which means that, after two iterations, the situation will be exactly the same as it was initially. In order to prevent the deoverlap process from going into an endless loop because of this, a time limit~(set to half a second by default) has been built in, after which the deoverlapping will stop.

%

\subsection{Parameterisation}
Two different implementations of this needed to be made, in order to find the best way of doing it. To allow for easy implementation of multiple interaction systems, a \verb|Behavior| interface has been created, which can be used to implement different behaviours~(see \figref{oframp_class}).

%

\subsection{Completing parameterisation}

\subsubsection{Downloading results}
\seclab{impl_downloading}
When the user feels he is completely done, he can download the resulting parameterisation of the molecule by clicking the `Download' button at the top of the page. This will give him the result in the \verb|LEMON Graph File|~(LGF) format~\cite{dezso2011lemon}. This format has been chosen as it is used by, among others, the fragment finding system~(see \secref{mop_fragments}), and because it is a flexible format that can easily be generated. This file will contain all atoms' element types, \verb|IACM| and charge, and all bonds between the atoms. It can currently not be imported back into the system, but support for that may be added at a later stage.

%

\subsection{Other features}
\subsubsection{Modifying visualisation parameters}
It has been implemented using a slight modification of the \verb|JavaScript| library \verb|dat-gui|~\cite{data2011dat}.

%

\section[\oapoc]{The Online tool for Atom Position Calculations}
The data that \verb|obabel| returns will be parsed and converted to a \verb|JSON| object, such that it can be sent to the client running \oframp. Additionally, \verb|IACM| atom types are calculated, following the algorithm presented by Malde~et.~al.~\cite{malde2011automated}. These atom types are needed later to find matching fragments for the molecule.

\subsection{From ATB}
When an \verb|ATB ID| is supplied to \oapoc, the molecule data string will be retrieved from the ATB. For all its molecules, the ATB stores a \verb|PDB| file, which is one of the formats that \oapoc{} supports. This file will be downloaded, and its contents will be used as the MDS for finding the molecule data.

\subsection{Open Babel}
\seclab{oapoc_obabel}
For converting molecule data strings from one format to another, it has been decided to go with \verb|Open Babel|~\cite{oboyle2011open}. This tool can be used easily and for free, and has support for most common MDS formats. Additionally, it performs quite well, and is known to provide good atom positions.

From \oapoc, the \verb|Open Babel| executable \verb|obabel| takes an MDS and the format that MDS uses. \verb|obabel| will then convert the MDS to the \verb|Mol2| format. This format has been chosen as it can easily be parsed and contains information on atom positions and bonds. Furthermore, it does not leave out the \verb|H| atoms, which some other formats do. For some applications this is fine, but for \oframp{} it is required to have all \verb|H| atoms.

%

\section[\omfraf]{The Online tool for Molecule Fragment Finding}
\subsection{Getting the repositories}
In order to be able to select a repository, the user needs to know what repositories are available. As can be seen in \figref{network_diagram}, a list of repositories can be retrieved by sending an empty \verb|JSON| object to \omfraf's \verb|/repos| URL. As of now, this list consists of only the repository titles. Later, this may be extended to also include the list of molecules that are contained in that repository, combined with functionality to exclude individual molecules. This will be discussed in more detail in \secref{futwork}.

\subsection{Generating fragments}
\seclab{impl_generating}
Before any fragments can be found, all matching fragments for the molecule need to be generated. As some complex chemical concepts need to be understood to determine whether a similar fragment is a good match, an external tool needs to be used for this. In \omfraf, fragments are generated using El-Kebir's \verb|fragments| tool, from the \verb|mop| project~\cite{elkebir2014molecule}.

As can be seen in \figref{network_diagram}, the \verb|fragments| tool has been wrapped in a program called \verb|fragment_generator|. For every molecule in the provided repository, this tool will retrieve the matching fragments with the input molecule, transform them into a slightly more manageable format, combine the fragments of all molecules, and store them to disk in the so-called \omfraf{} Fragments File~(OFF) format. The name of this file will be sent back to the \oframp{} user, such that he can query for fragments later. Along with the fragment file's name, a list of atoms for which no matching fragments could be found is sent to the user, which will allow for indicating these atoms in the molecule visualisation~(see \secref{impl_completing}).

%

\subsection{Finding fragments}
\seclab{impl_finding}
Once the fragments have been generated, the user of \oframp{} will be able to start finding fragments. The selected atom(s) and OFF filename will be sent to \omfraf, which will invoke the \verb|fragment_finder|. This program will load the fragments from the OFF file, and select those in which \emph{all} selected atoms are present. Those fragments will be ordered based on the score that was assigned to them by the \verb|fragments| tool, and are returned to the \oframp{} front-end such that they can be used in the parameterisation process.
