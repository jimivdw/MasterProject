\chapter{Discussion}
\chlab{discussion}

From the results obtained in the user studies, it can be decided which version of \oframp{} is the best for the task of fragment-based molecule parameterisation. What version that is will be argued in this chapter, along with a description of how it can be improved in the future. Additionally, some threats to the validity of the experiment outcomes will be discussed.


\section{The best version}
There are several factors that can help to determine which interaction design for \oframp{} is the best. First, the time required to complete a parameterisation an be considered. As discussed in \secref{res_time}, the smart version requires the least time per atom, and will therefore be the fastest to completely parameterise a molecule. With a total median time of under five minutes for parameterising all four molecules in the user study, it can also be considered to be fast in general. Time-wise, the smart version is therefore the best.

Second, one can look at the correctness of the resulting parameterisations of the different versions. The results presented in \secref{res_rating} show that the naive version has the lowest overall charge difference for both the first and second set of molecules. Furthermore, it has a lower average atom charge difference for every parameterised molecule as well, with average differences of $0.054$ for the smaller molecules, and $0.146$ for the bigger ones. This means that, result-wise the naive version is the best.

Blah blah blah\ldots


\section{Threats to validity}
\nlipsum


\section{Future work}
\nlipsum
