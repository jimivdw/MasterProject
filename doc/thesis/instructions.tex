\chapter{User studies instructions}
\chlab{instructions}

In this chapter, the exact instructions that have been provided to the user studies participants can be found. Note that these are the instructions for the group that first did the consuming version of \oframp, followed by the producing version. The other group of participants received similar instructions, but just with the URLs swapped.

Please execute the following steps in the exact order in which they are presented:
\begin{enumerate}
\item Go to OFraMP at \url{http://vps955.directvps.nl/OFraMP/n/};
\item As soon as the page has been loaded, click the `Start demo' button;
\item Follow the instructions given to you by the demo, and finish the parameterisation of the demo molecule (does not need to be perfect, just make sure every atom has a charge assigned);
\item Now, either reload the page or click the `New molecule' button at the top of the page;
\item Insert the following into the text input field: \textbf{10321}. This number represents an ATB ID, which we use here as it is easier to insert than a SMILES or InChI string. It represents 1,3-propandiol (PDO), a molecule with a total net charge of \textbf{0.0};
\item Make sure the repository is set to \textbf{arbitrary\_qm1} (an arbitrary selection of QM1 molecules from the ATB) and the shell size to \textbf{1}; then click the `Submit' button;
\item Fully parameterise the molecule, just like you did in the demo. This time, try to make it as good as possible and try to make the charges add up to the total net charge of the molecule. Feel free to consult the Help page if you need more detailed information;
\item \underline{Make sure to download the result LGF file} using either the `Download LGF' button in the `Fully parameterised' popup, or the `Download' button at the top of the page;
\item Now, enter a new molecule \textbf{16978} (\_V3M, total net charge of \textbf{-1.0}) and repeat steps 6 through 8. Again, \underline{remember to download the result file};
\item Please evaluate \emph{this version} of the system at \url{http://vps955.directvps.nl/OFraMP/nq/}. \underline{Try to focus as much as you can on the user interaction aspect}. Note that statements are ranked on a scale of 1 (strongly \emph{dis}agree) to 5 (strongly agree);
\item As soon as you have submitted the questionnaire, please go to the second version of OFraMP, available at \url{http://vps955.directvps.nl/OFraMP/s/};
\item Quickly take the demo again, in order to get to know this version and the differences with the other;
\item Fully parameterise the molecule \textbf{13913} (1h-imidazol-2-amine (2AI), total net charge of \textbf{0.0}). Again, make sure the repository and shell size are set to the values shown in step 6, and please \underline{download the result file as soon as you are done};
\item Parameterise one final molecule: \textbf{17738} (borrelidin (\_VOQ), total net charge of \textbf{-1.0}), with the same repository and shell size, and also \underline{download this one's result};
\item Please evaluate \emph{this version} of the system at \url{http://vps955.directvps.nl/OFraMP/sq/}. Again, \underline{try to focus as much as you can on the user interaction aspect}. Note that statements are, again, ranked on a scale of 1 (strongly \emph{dis}agree) to 5 (strongly agree);
\item Finally, please answer some general questions at \url{http://vps955.directvps.nl/OFraMP/q/}.
\end{enumerate}
Good luck, and thanks again for participating in the user studies.

