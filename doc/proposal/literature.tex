\chapter{Literature survey}
\chlab{literature}

For this research project, quite a lot of literature is available. The literature has been organised into four different categories, for each of which a number of papers will be discussed.


\section{Molecule simulations}
\seclab{simulations}

Canzar, El-Kebir, et al. discuss their work on charge group partitioning~\cite{canzar2012charge}. They explain what is needed for running biomolecular simulations and in which steps this information can be gathered. When the atom types, bonds and partial charges of a molecule are known, calculating the charge groups allows for running biomolecular simulations on very small time scales, while still being precise. $\mathcal{NP}$-hardness of the partitioning problem is proven, and an algorithm for quickly calculating the charge groups is presented. The proposed research project is not about assigning charge groups, but the work of Canzar and El-Kebir shows the need for finding the atomic partial charges of a molecule. It also indicates where in the process the partial charges should be calculated, and what will be done with them afterwards.

Malde et al. present the Automated Force Field Topology Builder~(ATB)~\cite{malde2011automated}. The ATB is a web service that can provide topologies for use in molecular simulations. It can both act as a repository for already parameterised molecules and it is able to parameterise molecules itself. This automatic parameterisation involves quantum mechanical calculations, which are very complex and computationally intensive. In the proposed research project, an attempt is made to improve the calculations of atomic partial charges. The results obtained by the tool to be developed will be compared to those obtained by the ATB, in order to validate the value of new tool.


\section{Interaction design}
\seclab{design}

Besides trying to optimise the preparation steps for molecule simulations, the main challenge of this research project is to come up with a good interaction design for a tool that does that. In the interaction design area, Donald Norman is a respected and often cited author. He feels that technology can only live up to its full potential by, first of all, supporting human tasks, while making the supporting technology as transparent as possible by making the tools easy to use, easy to learn and easy to understand. Designing for this purpose is called user-centred design~\cite{norman2002design}.

One of the most important aspects of design is visibility. Every interface should have visible features that can send the right messages to the user. It is very important that user actions do not have coincidental consequences. Otherwise, the user can develop wrong expectations of his actions, which may later result in problems while using the tool. What is also important, is that if a user \emph{does} make a mistake, he should be able to undo this. If this is \emph{not} possible, he may get easily frustrated and will stop using the tool. Finally, the user should not get lost in an enormous list of features. As many of these features will often rarely be used, tools with less features are often the better ones.

An often occurring problem with interfaces is that they get in the way of the task that needs to be performed~\cite{norman1990interfaces}. The user of a tool should not be spending his time `using the interface', but rather on performing the task the tool is supposed to help him with. In order to achieve this, product design should start with analysing the user and the task, and only then designing the user interaction.

When doing interaction design, one has to keep in mind that the appearance of a tool can have a great impact on how well the user of that tool can carry out the tasks the tool is intended for~\cite{norman2002emotion}. Tools that look unattractive tend to focus the mind, which leads to better concentration. For tasks where something needs to be done quickly, this is good, but in cases where creative thinking is required, this will not provide a satisfying result. In that case, the thought processes should be broadened by having something that looks really good. This can easily get the user distracted, which allows for creative thinking.

In an attempt to overcome the poor design of many software products, an approach called human-centred design (HCD) has been developed~\cite{norman2005human}. However, even though the needs of the user play a central role here, many companies following the HCD principles still develop complex and confusing products. The systems are often superb at the level of the static, individual display, but fail to support the sequential requirements of the underlying tasks and activities. Therefore, a new approach is needed that does not put the user in a central role, but rather does so with the activity that needs to be performed. In this activity-centred design method, it is sometimes necessary to ignore a user's requests as they might compromise the task that has to be carried out.

This, however, does not mean that the application user's needs should be completely ignored. On the contrary, it is really important that the user's characteristics are understood and used in the design of an application~\cite{badre2002shaping}. Also, a good analysis of the context is needed to make the right design decisions. If one ignores the user characteristics and context of an application, the application is  likely to fail, as the context in which the application is developed will almost always differ from the context in which it will be used.

Following the previously discussed and other research, a number of principles of interaction design can be identified. However, different authors have different opinions on what those principles are. Norman and Nielsen identify the following principles of interaction design that are completely independent of technology~\cite{norman2010gestural}:
\begin{itemize}[noitemsep,topsep=0pt,parsep=0pt,partopsep=0pt]
\item Visibility (also called perceived affordances or signifiers);
\item Feedback;
\item Consistency (also known as standards);
\item Non-destructive operations (hence the importance of undo);
\item Discoverability: all operations can be discovered by systematic exploration of menus;
\item Scalability: the operation should work on all screen sizes, small and large;
\item Reliability: operations should work and events should not happen randomly.
\end{itemize}
Thimbleby, on the other hand, \emph{does} relate his principles to technology~\cite{thimbleby2007press}:
\begin{itemize}[noitemsep,topsep=0pt,parsep=0pt,partopsep=0pt]
\item Use good algorithms for better user interfaces: interactive devices are designed to solve problems in a structured way, which is exactly what a good algorithm does;
\item Use simple, explicit interaction frameworks: those provide clear interaction structures, which allow for reliable feature integration, checking, analysis, fault identification, and error fixing;
\item Interrelate all interaction programming: all aspects of the design should come out of the same specification;
\item Define properties and analyse designs for them.
\end{itemize}
A more extensive list is provided by Blair-Early and Zender~\cite{blair2008user}:
\begin{itemize}[noitemsep,topsep=0pt,parsep=0pt,partopsep=0pt]
\item Obvious start: design an obvious starting point;
\item Clear reverse: design an obvious exit or stop;
\item Consistent logic: design an internally consistent logic for content, actions, and effects (the most important consistency is that with user expectations);
\item Observe conventions: identify and consider the impact of familiar interface conventions;
\item Feedback: design tangible responses to apt user actions;
\item Landmarks: design landmarks as a reference for context;
\item Proximity: design interface elements in consistent proximity to their content objects and to each other;
\item Adaptation: design an interface that adapts or is adapted to use;
\item Help: as necessary, provide a readily accessible overall mechanism for assistance;
\item Interface is content: design interface elements that minimise interface and maximise content.
\end{itemize}


\section{Molecule software}
\seclab{software}

There is a lot of software available for displaying, drawing and editing molecules. These programs form an indispensable part of every molecular processing system~\cite{ertl2010molecular}. Throughout the years, these programs have evolved from basic text editors, through clickable image maps, to full-on molecular structure sketching software. In the last few years, a new trend becomes visible. More and more cheminformatics applications are being brought to the web, allowing for a lot of new ways of interaction. Furthermore, these new applications are being open sourced, allowing for new innovative variations or combinations of the old tools.

An example of a previously offline, closed source tool is \verb|JSmol|, previously known as just \verb|Jmol|~\cite{hanson2013jsmol}. This tool has been seamlessly transformed from a \verb|Java| applet to \verb|JavaScript|, without any visible visual difference. Furthermore, performance wise, there is only a minor difference between the two implementations. Another example is \verb|JSME|~\cite{bienfait2013jsme}. This tool has even been cross-compiled from its original \verb|Java| code to \verb|JavaScript| using the \verb|Google Web Toolkit| compiler. The transition to the web has resulted in the addition of several features, suggested by users from the new, bigger audience. What can be concluded from these two cases is that bringing molecule software to the web opens up a whole world of new possibilities, without having to give up on performance.

With the ongoing migration of molecule software to the web, new devices, such as smartphones and tablets, will be able to run the tools. As these devices promote different ways of user interaction and often have smaller screens, designs of the molecule software need to be reconsidered. \verb|TB Mobile| is a mobile app for identification of potential anti-tuberculosis molecules~\cite{ekins2013tb}. The app is structured in such way that there is always a small control bar at the top, and a large area for showing the content. Interaction with the app is handled by a small number of large buttons, allowing for easy touch controls. In order to work for different screen sizes, everything in the content area is scalable, and, in case not everything fits on the screen, scrollable. Usage of the app in practice has shown that it helps to improve the work flow of tuberculosis researchers, by lowering the barriers for accessing the information it provides.

The issue of showing a lot of data on a limited-size screen is also addressed by Ertl and Rohde~\cite{ertl2012molecule}. They took the concept of a word cloud, and transformed that to a molecule cloud, where the highest scored molecules are the biggest and the lowest are the smallest. None of the molecules overlap, and the sizes are divided such that the whole available screen space is filled. User studies have shown that these molecule clouds provide an easy way of finding the most relevant molecules. In the proposed research project, this may be useful for showing the set of related fragments of other molecules, depending on the size of that set.


\section{User studies}
\seclab{user_studies}

In Software Engineering, an often used method to evaluate a project is performing a user study. In general, user studies can be assigned to one of the following categories: experiments, case studies, surveys, and post-mortem analyses~\cite{wohlin2003empirical}. In all of these types of studies, it is important to representative group of test subjects. These should preferably be picked at random from the population of application users.

This project, first of all, will subject a number of users to an experiment. In an experiment, two (or more) different configurations of the examined tool are compared by examining the effects the configuration has on the tool's output~\cite{wohlin2003empirical}. It is important to define exactly what one wants to validate~\cite{stein2009assessing}. This should preferably be something that can easily be observed (e.g. time required or number of clicks, but not comprehensibility). Furthermore, it is important that all different configurations are semantically equivalent (i.e. doing the same thing), have an equal degree of compression (i.e. contain the same information) and formatted into their cleanest and clearest extent.

Before starting an experiment, it is very important to make sure your subjects are committed to the tasks they need to perform. Otherwise, the experiment will not properly reflect the production environment. Furthermore, the experiment should be well-prepared, as start-up problems might negatively impact the subject's opinion otherwise.

Besides the experiment, this project will also contain a survey. In a survey, it is easy to test a large number of variables, but one should be careful not to test too many, as this will make the analysis infeasible~\cite{wohlin2003empirical}. In order to collect questionnaire information in a structured way, its questions should follow some natural flow that embodies all aspects of a user's experience~\cite{tuch2013analyzing}. This encourages them to report their detailed experiences, as this is less demanding than answering one big open question. Furthermore, it also reduces the chances of subjects failing to report certain things, simply because they forgot about them.

A few predefined surveys are available for usability testing, including the System Usability Scale~(SUS) and the Usability Metric for User Experience~(UMUX)~\cite{lewis2013umux}. The SUS consists of ten items, each of which should be graded on a scale of 1 (strongly disagree) to 7 (strongly agree). It has been extensively tested and turns out to have a reliability score of 0.96. UMUX is a shorter questionnaire, consisting of four questions, but its even shorter brother, UMUX-LITE, consisting of only two questions, has been shown to provide results of the same reliability as UMUX. When a system is subjected to both SUS and UMUX-LITE, one can get very reliable results on the usability of that system.

On a final note about surveys, the fact that they are often straight-forward makes them perfectly suitable for being held over the internet. It has been shown that the results of online surveys do not significantly differ from their offline counterparts~\cite{komarov2013crowdsourcing}. However, face-to-face surveys are still the only way that allows both the interviewer and the test subject to immediately ask and answer questions~\cite{wohlin2003empirical}. Even if this does not lead to a significant difference in the outcome of the survey, it can definitely lead to new insights.

Before analysing the data obtained by a user study, it is important to validate it~\cite{wohlin2003empirical}. All the obtained results should be complete and correctly documented. Furthermore, extreme outliers should be identified dealt with. What should happen exactly depends on the situation, but in most cases it is best to leave out the extreme results~\cite{komarov2013crowdsourcing}. Finally, it is very important to provide a grade of validity along with the conclusions of the user studies. This grade depends on, for instance, the number of test subjects, their representativeness or the distribution of the results.


\section{Molecule data formats}
\seclab{data_formats}

For importing, storing, and exchanging molecule structure data, various formats exist. One of these formats is \verb|SMILES|, a very minimal line notation for molecule structure data~\cite{daylight1992daylight}. Back in 1988, when \verb|SMILES| was first introduced, it was quite common to store molecular structures as connection tables. These tables took a lot of storage space and were not easily exchangeable, so something new needed to be developed. The \verb|SMILES| representation of a molecule is just a simple linguistic construct that uses a simple vocabulary for atom an bond symbols. It has remained popular since then, and is still used in many molecule editors today.

Nevertheless, attempts have been made to replace the \verb|SMILES| format, as it is a closed standard with only limited possibilities. In 2005, a new format called \verb|IUPAC InChI| was introduced~\cite{heller2013inchi}. \verb|InChI| is an open format that aims to combine chemical, biological and other related information in a single representation. Still being a linguistic representation like \verb|SMILES|, it is easily exchangeable and allows for simple machine processing. In the past few years, \verb|InChI| has become the international, worldwide standard for defined chemical structures. As \verb|SMILES| is also still widely used, the tool that will be developed in the proposed research project will support both the \verb|SMILES| and \verb|InChI| formats for entering molecules into the system.
