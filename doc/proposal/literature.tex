\chapter{Literature survey}
\chlab{literature}

For this research project, quite a lot of literature is available. The literature has been organised into four different categories, for each of which a number of papers will be discussed.


\section{Molecule simulations}
\seclab{simulations}

Canzar and El-Kebir discuss their work on charge group partitioning~\cite{canzar2012charge}. They explain what is needed for running biomolecular simulations and in which steps this information can be gathered. When the atom types, bonds and partial charges of a molecule are known, calculating the charge groups allows for running biomolecular simulations on very small time scales, while still being precise. $\mathcal{NP}$-hardness of the partitioning problem is proven, and an algorithm for quickly calculating the charge groups is presented. This research project is not about assigning charge groups, but the work of Canzar and El-Kebir shows the need for finding the atomic partial charges of a molecule. It also indicates where in the process the partial charges should be calculated, and what will be done with them afterwards.

Malde et al. present the Automated Force Field Topology Builder~(ATB)~\cite{malde2011automated}. The ATB is a web server that can provide topologies for use in molecular simulations. It can act as both a repository for already parameterised molecules and can also parameterise molecules itself. Etc. etc.


\section{Interaction design}
\seclab{design}

Papers ~\cite{norman2002design}, ~\cite{norman1990interfaces}, ~\cite{norman2002emotion}, ~\cite{norman2005human}, ~\cite{norman2010gestural}, ~\cite{thimbleby2007press}, ~\cite{blair2008user}, ~\cite{badre2002shaping}


\section{Molecule software}
\seclab{software}

Papers ~\cite{ertl2010molecular}, ~\cite{hanson2013jsmol}, ~\cite{bienfait2013jsme}, ~\cite{ekins2013tb}, ~\cite{fjeld2007tangible}, ~\cite{ertl2012molecule}


\section{Molecule data formats}
\seclab{data_formats}

Papers ~\cite{daylight1992daylight}, ~\cite{heller2013inchi}





Describe the following for related scientific literature on your topic:
\begin{itemize}
\item Brief summary of contents
\item Relation to your topic: how does the work described in the reference differ from your approach, what results have they obtained, what open questions do they still have?
\end{itemize}
