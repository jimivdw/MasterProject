\chapter{Literature survey}
\chlab{literature}

For this research project, quite a lot of literature is available. The literature has been organised into four different categories, for each of which a number of papers will be discussed.


\section{Molecule simulations}
\seclab{simulations}

Canzar and El-Kebir discuss their work on charge group partitioning~\cite{canzar2012charge}. They explain what is needed for running biomolecular simulations and in which steps this information can be gathered. When the atom types, bonds and partial charges of a molecule are known, calculating the charge groups allows for running biomolecular simulations on very small time scales, while still being precise. $\mathcal{NP}$-hardness of the partitioning problem is proven, and an algorithm for quickly calculating the charge groups is presented. This research project is not about assigning charge groups, but the work of Canzar and El-Kebir shows the need for finding the atomic partial charges of a molecule. It also indicates where in the process the partial charges should be calculated, and what will be done with them afterwards.

Malde et al. present the Automated Force Field Topology Builder~(ATB)~\cite{malde2011automated}. The ATB is a web service that can provide topologies for use in molecular simulations. It can both act as a repository for already parameterised molecules and it is also able to parameterise molecules itself. This automatic parameterisation is done using some quantum mechanical calculations, which are very complex and computationally intensive. In this research project, an attempt is made to improve the calculations of atomic partial charges. The results obtained by the tool to be developed will be compared to those obtained by the ATB, in order to validate the value of new tool.


\section{Interaction design}
\seclab{design}

Besides trying to optimise the preparation steps for molecule simulations, the main challenge of this research project is to come up with a good interaction design for a tool that does that. In the interaction design area, Donald Norman is a respected and often cited author. He feels that technology can only live up to its full potential if, first of all, supporting human tasks, while making the supporting technology as transparent as possible by making the tools easy to use, easy to learn and easy to understand. Designing for this purpose is called user-centred design~\cite{norman2002design}.

One of the most important aspects of design is visibility. Every interface should have visible features that can send the right messages to the user. It is very important that a user's actions do not have coincidental consequences. Otherwise, the user can develop wrong expectations of his actions, which may later result in problems while using the tool. What is also important, is that if a user does make a mistake, he should be able to undo this. If this is not possible, he may get easily frustrated and will stop using the tool. Finally, the user may not get lost in an enormous list of features. As many of these features will often rarely be used, tools with less features are often the better ones.

An often occurring problem with interfaces is that they get in the way of the task that needs to be performed~\cite{norman1990interfaces}. The user of a tool should not be spending his time 'using the interface', but rather on performing the task the tool is supposed to help them with. In order to achieve this, product design should start with analysing the user and the task, and only then designing the user interaction.

When doing interaction design, one has to keep in mind that the appearance of a tool can have a great impact on how well the user of that tool can carry out the tasks the tool is intended for~\cite{norman2002emotion}. Tools that look unattractive tend to focus the mind, which leads to better concentration. For tasks where something needs to be done quickly, this is good, but in cases where creative thinking is required, this will not provide a satisfying result. In that case, the thought processes should be broadened, meaning that one can get easily distracted, by having something that looks really good.


, ~\cite{norman2005human}, ~\cite{norman2010gestural}, ~\cite{thimbleby2007press}, ~\cite{blair2008user}, ~\cite{badre2002shaping}


\section{Molecule software}
\seclab{software}

Papers ~\cite{ertl2010molecular}, ~\cite{hanson2013jsmol}, ~\cite{bienfait2013jsme}, ~\cite{ekins2013tb}, ~\cite{fjeld2007tangible}, ~\cite{ertl2012molecule}


\section{Molecule data formats}
\seclab{data_formats}

Papers ~\cite{daylight1992daylight}, ~\cite{heller2013inchi}





Describe the following for related scientific literature on your topic:
\begin{itemize}
\item Brief summary of contents
\item Relation to your topic: how does the work described in the reference differ from your approach, what results have they obtained, what open questions do they still have?
\end{itemize}
