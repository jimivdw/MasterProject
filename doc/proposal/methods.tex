\chapter{Research methods}
\chlab{methods}

The main question that this research project will try to answer is the following:
\begin{quote}
How can chemists interact with a tool for fragment-based molecule parameterisation, such that this yields results of comparable quality to conventional methods, while being done faster?
\end{quote}
In order to answer this question, such tool will be designed and a prototype of it will be implemented. User studies will be conducted to evaluate the tool's design and results. In those studies, the results obtained by using the tool will be compared to those obtained using conventional methods, i.e. complex quantum mechanical calculations~(see~\chref{problems}).


\section{The tool}
\seclab{tool}

The requirements of the molecule parameterisation tool have been specified in \chref{problems}. The decision has been made to implement it in \verb|HTML5| and \verb|JavaScript|, which allows for great portability and availability across different operating systems and platforms. It also makes the tool future-proof, by following the current trend of bringing everything to the web.

Not surprisingly, there are no existing tools for fragment-based molecule parameterisation, as this is a new concept. Furthermore, no tools exist for comparing molecules - or fragments of them - either. What does exist is a wide range of tools and programs for showing and editing molecules. This includes stand-alone molecule drawing software such as \verb|Accelerys Draw|~\cite{accelrys2012accelrys} and \verb|Avogadro|~\cite{hanwell2012avogadro}, two-dimensional web-based molecule editors like \verb|ChemDoodle 2D Sketcher|~\cite{ichemlabs2013chemdoodle}, \verb|Molsoft HTML5 Molecule Editor|~\cite{molsoft2012molsoft} and \verb|Marvin for JavaScript|~\cite{chemxon2013marvin}, and online three-dimensional visualisation tools \verb|JSMol|~\cite{hanson2013jsmol} and \verb|CanvasMol|~\cite{altered2013canvasmol}.

These existing tools will serve as an initial guideline for the tool to be developed, and parts of their implementations may be reused. The rest of the tool, however, will need to be designed and developed from scratch. The design will follow the basic interaction design principles as posed by Norman and others~(see \secref{design}), and keep in mind the insights gained by the developers of other molecule software~(see \secref{software}).

% TODO: expand + add reference back to ID challenges in chapter 2
Insert more concrete plan of approach w.r.t. interaction design here...


\section{Evaluation}
\seclab{evaluation}

% TODO: check on TODOs
To evaluate the project, the developed tool will be subjected to a number of user studies. Currently, however, the presumed user base is limited to a small number of researchers. Furthermore, half of these researchers is located in Australia, which makes it infeasible to include them in user studies. Luckily, project supervisors Klau and El-Kebir keep a close connection with most other target researchers at VU University Amsterdam. They will presumably be willing to participate in the user studies and can hopefully help to find some other test subjects$^{[TODO!!!]}$.

In the user studies, the test subjects will be asked to parameterise a few molecules of increasing size and complexity. The first one will be quite easy to parameterise, but the last one should be of such complexity that the conventional quantum mechanical calculations start struggling and really taking up time. It is in these last situations that the tool should show its true value.

During the tasks, the time required to complete the parameterisation will be measured. There are no existing comparable tools, so the timing results cannot be compared to some baseline, but completing the parameterisation should definitely not require more than XX minutes$^{[TODO!!!]}$. Furthermore, users should never be annoyed or willing to stop parameterising.

Besides time, the user will also be scored on his performance. In order to do this, the molecules that the user will be asked to parameterise should already have been parameterised using the conventional quantum mechanical method. This way, the parameterisation of the user and the calculations can be compared. The smaller the difference, the better the performance of the user will be scored. Of course, there will always be a small difference between the manual and automatic ways, as the manual assignment cannot be as precise as the automatic one is, but as long as the difference is small, the developed tool can be really useful to speed up atomic charge assignment.

After completing their tasks, the users will be asked to answer a number of questions about the tool. These will mainly be about how they like the design and if they can see themselves using it, but will also ask them for suggestions on things that can be improved or added. This way, their experiences can be used to further improve the tool, and to make it a tool they really like using.


\section{Time line}
\seclab{timeline}

There is a period of five months available for this project. These five months will be divided as follows:

\noindent
\begin{tabular}{r|l}
\textbf{Sep} & Project plan and literature study\\
\textbf{Oct-Dec} & Iteratively design, implement and validate the tool\\
\textbf{Jan} & Thesis\\
\textbf{Feb} & Thesis defence
\end{tabular}

\noindent
Note that in September and October, the project will only be worked on three days per week. From November to January, the project will be worked on full time.

User studies with real potential application users will be held in the first half of December. This allows for an additional period in the implementation phase in which any flaws discovered by the user studies can be fixed. It also means there should be a working version of the tool by then. In the period of the implementation phase up to December, frequent progress meetings with the project supervisors will be organised, to make sure flaws are discovered at an early stage and can be quickly solved. This takes away the need for a testing, fixing and improving phase at the end of the project.
