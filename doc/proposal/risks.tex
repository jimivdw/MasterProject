\chapter{Risks}
\chlab{risks}

Just like every other research project, this project does not come without risks. In this chapter, the risks that come with this project will be discussed, along with possible counter measures.

\begin{description}
\item[Limited knowledge about chemistry]~~\\
It does not seem that thorough understanding of chemistry is really needed for this project. If, however, it turns out that at some point more in-depth chemistry knowledge is needed, the supervisors of this project have agreed to help solving the chemistry related problems.

\item[Molecule fragment matching algorithm is still being developed]~~\\
Besides repeatedly asking people to finish something, there is not much you can do to make sure something others are developing will be done in time. A workaround for this problem would be to temporarily use a fixed, static set of molecule fragments. These fragments may or may not be related to the molecule that is being analysed, but al least this allows for almost complete development of the application without being dependent on the work of others.

\item[Existing tools may not be easily extendible]~~\\
If the existing molecule drawing applications turn out to be unextendible, the visualisation of the molecules will have to be implemented from scratch. Luckily, there are some open source graph drawing libraries for \verb|JavaScript|, e.g. \verb|D3.js|~\cite{bostock2012data}, \verb|Raphael.js|~\cite{baranovski2013raphael} and \verb|JavaScript Graph Library|~\cite{dracula2012javascript}. These libraries reduce the complexity from having to implement a whole drawing system to restricting an existing drawing system such that it will correctly draw molecules. 

\item[No real comparable tools exist]~~\\
As there are no existing tools for fragment-based molecule parameterisation, it is unknown what will be a good way of doing this. To mitigate this risk, frequent meetings with the project supervisors will be organised to ensure flaws are discovered in an early phase.

\item[Small initial target audience]~~\\
Initially, there will only be a small group of potential users for the application. This limits the number of possible subjects for the user studies, which might pose a threat for the evaluation process. In order to mitigate this risk, user studies will not take place too early in the development process, but only as soon as the tool seems to be complete. Furthermore, as discussed in~\secref{evaluation}, some students will be asked to evaluate the tool as well. This should leave a big enough group of test users to be able to say something about the quality of the tool.

\item[Focussing on the tool rather than the research]~~\\
When a research project involves the development of a tool it is always tempting to put a lot of time into developing that tool, while completely forgetting about the research. To make sure that does not happen in this project, there will be frequent contact with the project supervisors to ensure enough research is being done. Furthermore, initial development of the tool will be stopped as soon as a point is reached where it does everything it is supposed to do, even if there still is a lot of time left at that point. This makes sure there is enough time to properly validate the tool and make this into a good research project.
\end{description}
