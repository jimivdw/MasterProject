\chapter{Project summary}
\chlab{summary}

In the world of biomolecular science, molecular simulations are gaining in importance. These simulations are used to study biomolecular systems at an atomic level such that interactions between molecules can be analysed. To perform these simulations, a set of parameters is needed, including the atomic partial charges of a molecule. Calculating these charges, however, can take a lot of time, even on advanced computation clusters.

As it is seemingly impossible to speed up the partial charge calculations, an alternative way to retrieve them should be developed. It is known that similar sections in two different molecules often have roughly the same atomic charges. Therefore, it should be possible to assign the charges of a molecule based on the known charges of sections of other molecules.

Finding the best matches is something only humans (read: experienced scientists) can do. Therefore, a tool needs to be designed that allows them to visualise a molecule and a set of related sections of other molecules. They should be able to pick the best matches and assign their atomic charges to the molecule that is being analysed. The tool should also allow for manual adjustment of the charges, as these are never exactly the same in other molecules.

This project aims to answer the following research questions:
\begin{itemize}
\item How can chemists interact with a tool for assigning partial atomic charges, based on those of similar sections in other molecules?
\item Is it feasible to do manual partial charge assignment in a reasonable amount of time?
\item Compared to computed partial charges, how well can manual atomic charge assignment be done?
\end{itemize}
